\documentclass[journal,12pt,twocolumn]{IEEEtran}
%
\usepackage{setspace}
\usepackage{gensymb}
\singlespacing
\usepackage[cmex10]{amsmath}
\usepackage{siunitx}
\usepackage{amsthm}

\usepackage{mathrsfs}

\usepackage{txfonts}
\usepackage{stfloats}
\usepackage[table]{xcolor}
\usepackage{steinmetz}
\usepackage{cite}
\usepackage{cases}
\usepackage{subfig}
\usepackage{longtable}
\usepackage{multirow}
\usepackage{enumitem}
\usepackage{mathtools}
\usepackage{tikz}
\usepackage{circuitikz}
\usepackage{verbatim}
\usepackage{tfrupee}
\usepackage[breaklinks=true]{hyperref}
\usepackage{tkz-euclide} % loads  TikZ and tkz-base
\usetikzlibrary{calc,math}
\usetikzlibrary{fadings}
\usepackage{listings}
    \usepackage{color}                                            %%
    \usepackage{array}                                            %%
    \usepackage{longtable}                                        %%
    \usepackage{calc}                                             %%
    \usepackage{multirow}                                         %%
    \usepackage{hhline}                                           %%
    \usepackage{ifthen}                                           %%
  %optionally (for landscape tables embedded in another document): %%
    \usepackage{lscape}     
\usepackage{multicol}
\usepackage{chngcntr}
\usepackage{cancel}
\DeclareMathOperator*{\Res}{Res}

\renewcommand\thesection{\arabic{section}}
\renewcommand\thesubsection{\thesection.\arabic{subsection}}
\renewcommand\thesubsubsection{\thesubsection.\arabic{subsubsection}}

\renewcommand\thesectiondis{\arabic{section}}
\renewcommand\thesubsectiondis{\thesectiondis.\arabic{subsection}}
\renewcommand\thesubsubsectiondis{\thesubsectiondis.\arabic{subsubsection}}

\hyphenation{op-tical net-works semi-conduc-tor}
\def\inputGnumericTable{}                                 %%

\lstset{
%language=C,
frame=single, 
breaklines=true,
columns=fullflexible
}
\begin{document}
%


\newtheorem{theorem}{Theorem}[section]
\newtheorem{problem}{Problem}
\newtheorem{proposition}{Proposition}[section]
\newtheorem{lemma}{Lemma}[section]
\newtheorem{corollary}[theorem]{Corollary}
\newtheorem{example}{Example}[section]
\newtheorem{definition}[problem]{Definition}
\newcommand{\BEQA}{\begin{eqnarray}}
\newcommand{\EEQA}{\end{eqnarray}}
\newcommand{\define}{\stackrel{\triangle}{=}}
\bibliographystyle{IEEEtran}
\providecommand{\mbf}{\mathbf}
\providecommand{\pr}[1]{\ensuremath{\Pr\left(#1\right)}}
\providecommand{\qfunc}[1]{\ensuremath{Q\left(#1\right)}}
\providecommand{\sbrak}[1]{\ensuremath{{}\left[#1\right]}}
\providecommand{\lsbrak}[1]{\ensuremath{{}\left[#1\right.}}
\providecommand{\rsbrak}[1]{\ensuremath{{}\left.#1\right]}}
\providecommand{\brak}[1]{\ensuremath{\left(#1\right)}}
\providecommand{\lbrak}[1]{\ensuremath{\left(#1\right.}}
\providecommand{\rbrak}[1]{\ensuremath{\left.#1\right)}}
\providecommand{\cbrak}[1]{\ensuremath{\left\{#1\right\}}}
\providecommand{\lcbrak}[1]{\ensuremath{\left\{#1\right.}}
\providecommand{\rcbrak}[1]{\ensuremath{\left.#1\right\}}}
\theoremstyle{remark}
\newtheorem{rem}{Remark}
\newcommand{\sgn}{\mathop{\mathrm{sgn}}}
\providecommand{\abs}[1]{\left\vert#1\right\vert}
\providecommand{\abs}[1]{\lvert#1\rvert} 
\providecommand{\res}[1]{\Res\displaylimits_{#1}} 
\providecommand{\norm}[1]{\left\lVert#1\right\rVert}
%\providecommand{\norm}[1]{\lVert#1\rVert}
\providecommand{\mtx}[1]{\mathbf{#1}}
\providecommand{\mean}[1]{E\left[ #1 \right]}
\providecommand{\fourier}{\overset{\mathcal{F}}{ \rightleftharpoons}}
%\providecommand{\hilbert}{\overset{\mathcal{H}}{ \rightleftharpoons}}
\providecommand{\system}{\overset{\mathcal{H}}{ \longleftrightarrow}}
	%\newcommand{\solution}[2]{\textbf{Solution:}{#1}}
\newcommand{\solution}{\noindent \textbf{Solution: }}
\newcommand{\cosec}{\,\text{cosec}\,}
\providecommand{\dec}[2]{\ensuremath{\overset{#1}{\underset{#2}{\gtrless}}}}
\newcommand{\myvec}[1]{\ensuremath{\begin{pmatrix}#1\end{pmatrix}}}
\newcommand{\mydet}[1]{\ensuremath{\begin{vmatrix}#1\end{vmatrix}}}
\numberwithin{equation}{subsection}
\makeatletter
\@addtoreset{figure}{problem}
\makeatother
\let\StandardTheFigure\thefigure
\let\vec\mathbf
\renewcommand{\thefigure}{\theproblem}
\def\putbox#1#2#3{\makebox[0in][l]{\makebox[#1][l]{}\raisebox{\baselineskip}[0in][0in]{\raisebox{#2}[0in][0in]{#3}}}}
     \def\rightbox#1{\makebox[0in][r]{#1}}
     \def\centbox#1{\makebox[0in]{#1}}
     \def\topbox#1{\raisebox{-\baselineskip}[0in][0in]{#1}}
     \def\midbox#1{\raisebox{-0.5\baselineskip}[0in][0in]{#1}}
\vspace{3cm}
\title{ASSIGNMENT-12}
\author{R.YAMINI}
\maketitle
\newpage
\bigskip
\renewcommand{\thefigure}{\theenumi}
\renewcommand{\thetable}{\theenumi}


%
\section{QUESTION No-2.28(Optimization)}
Two godowns A and B have grain capacity of 100 quintals and 50 quintals respectively.They supply to 3 ration shops, D, E and F whose requirements are 60, 50 and 40 quintals respectively. The cost of transportation per quintal from the godowns to the shops are given in the following table:
\numberwithin{table}{section}
\begin{table}[!ht]
\begin{center}
\begin{tabular}{ |l|l|l|}
\hline
\multicolumn{3}{ |c| }{Transportation cost per quintal (in rupees)} \\
\hline
From/To & A & B \\ \hline
D & 6 & 4  \\ \hline
E & 3 & 2 \\ \hline
F & 2.50 & 3 \\ \hline
\end{tabular}
\end{center}
\caption{Transportation table}
\label{tab:table1}
\end{table}
How should the supplies be transported in order that the transportation cost is minimum? What is the minimum cost?
\section{Solution}
The given table can be written as follows,
\begin{tabular}{cc|c|c|c|c|l}
\cline{3-6}
& & \multicolumn{4}{ c| }{Destination} \\ \cline{3-6}
& & D & E & F & Supply \\ \cline{1-6}
\multicolumn{1}{ |c  }{\multirow{2}{*}{Supplier} } &
\multicolumn{1}{ |c| }{A} & 6 & 3 & 2.50 & 100 &     \\ \cline{2-6}
\multicolumn{1}{ |c  }{}                        &
\multicolumn{1}{ |c| }{B} & 4 & 2 & 3 & 50 &     \\ \cline{1-6}
\multicolumn{1}{ |c| }{Demand} & & 60 & 50 & 40 & 150 \\ \cline{1-6}
\end{tabular}
\\
We solve this problem by Vogel's approximation method (VAM)
\begin{enumerate}
    \item For each row (excluding the demand row) we find the least value and then the second least value and take the absolute difference of these two least values.We see that for row1 and row2 the penalty is 0.50 and 1 respectively.
    \item For each column (excluding the supply column) we find the least value and then the second least value and take the absolute difference of these two least values.We get 2,1, and 0.50 for column1,column2 and column3 respectively.
    \item We select the row or column with the maximum penalty and find cell that has least cost in selected row or column. Allocate as much as possible in this cell.We see that among all the penalties we have maximum penalty to be 2 (first column penalty).In that column minimum cost is 4, the maximum allocation for 4 is 50. We can cancel out row2 (corresponding to B).Thus we have,
\\
\begin{tabular}{cc|c|c|c|c|l}
\cline{3-6}
& & \multicolumn{4}{ c| }{Destination} \\ \cline{3-6}
& & D & E & F & Supply \\ \cline{1-6}
\multicolumn{1}{ |c  }{\multirow{2}{*}{Supplier} } &
\multicolumn{1}{ |c| }{A} & 6 & 3 & 2.50 & 100 &     \\ \cline{2-6}
\multicolumn{1}{ |c  }{}                        &
\multicolumn{1}{ |c| }{B} &\cellcolor{yellow} 4\textcolor{red}{(50)} & \cellcolor{yellow}2 & \cellcolor{yellow}3 &\cellcolor{yellow} \cancel{50} &     \\ \cline{1-6}
\multicolumn{1}{ |c| }{Demand} & & \cancel{60} 10 & 50 & 40 & 150 \\ \cline{1-6}
\end{tabular}
\\
    \item Similarly we find penalty for remaining row1 and columns 1,2 and 3.
         For row 1 it is 0.50, column1 it is 6,column2 it is 3 and column3 it is 2.50.
         Maximum among the penalty is 6 so we choose column1 and the minimum cost is 6 and the maximum required allocation is 10.Thus we have,
\\
\begin{tabular}{cc|c|c|c|c|l}
\cline{3-6}
& & \multicolumn{4}{ c| }{Destination} \\ \cline{3-6}
& & D & E & F & Supply \\ \cline{1-6}
\multicolumn{1}{ |c  }{\multirow{2}{*}{Supplier} } &
\multicolumn{1}{ |c| }{A} & \cellcolor{yellow} 6\textcolor{red}{(10)} & 3 & 2.50 & \cancel{100} 90 &     \\ \cline{2-6}
\multicolumn{1}{ |c  }{}                        &
\multicolumn{1}{ |c| }{B} &\cellcolor{yellow} 4\textcolor{red}{(50)} & \cellcolor{yellow}2 &\cellcolor{yellow} 3 & \cellcolor{yellow}\cancel{50} &     \\ \cline{1-6}
\multicolumn{1}{ |c| }{Demand}& & \cellcolor{yellow} \cancel{60} & 50 & 40 & 150 \\ \cline{1-6}
\end{tabular}
\\
\item Now similarly we do for the remaining row and columns.We get the penalty of row1 to be 0.50 and for column2 and column3 we have 3 and 2.50 respectively.
Among the penalties the maximum is 3 and we allocate 50.Thus we have,
\\
\begin{tabular}{cc|c|c|c|c|l}
\cline{3-6}
& & \multicolumn{4}{ c| }{Destination} \\ \cline{3-6}
& & D & E & F & Supply \\ \cline{1-6}
\multicolumn{1}{ |c  }{\multirow{2}{*}{Supplier} } &
\multicolumn{1}{ |c| }{A} & \cellcolor{yellow} 6\textcolor{red}{(10)} & 3\textcolor{red}{(50)} & 2.50 & \cancel{100} 40 &     \\ \cline{2-6}
\multicolumn{1}{ |c  }{}                        &
\multicolumn{1}{ |c| }{B} &\cellcolor{yellow} 4\textcolor{red}{(50)} & \cellcolor{yellow}2 &\cellcolor{yellow} 3 & \cellcolor{yellow}\cancel{50} &     \\ \cline{1-6}
\multicolumn{1}{ |c| }{Demand}& & \cellcolor{yellow} \cancel{60} & \cancel{50} & 40 & 150 \\ \cline{1-6}
\end{tabular}
\\
\item Now the remaining 40 is allocated to the cell with cost 2.50.
\\
\begin{tabular}{cc|c|c|c|c|l}
\cline{3-6}
& & \multicolumn{4}{ c| }{Destination} \\ \cline{3-6}
& & D & E & F & Supply \\ \cline{1-6}
\multicolumn{1}{ |c  }{\multirow{2}{*}{Supplier} } &
\multicolumn{1}{ |c| }{A} & \cellcolor{yellow} 6\textcolor{red}{(10)} & 3\textcolor{red}{(50)} & 2.50\textcolor{red}{(40)} & \cancel{100} &     \\ \cline{2-6}
\multicolumn{1}{ |c  }{}                        &
\multicolumn{1}{ |c| }{B} &\cellcolor{yellow} 4\textcolor{red}{(50)} & \cellcolor{yellow}2 &\cellcolor{yellow} 3 & \cellcolor{yellow}\cancel{50} &     \\ \cline{1-6}
\multicolumn{1}{ |c| }{Demand}& & \cellcolor{yellow} \cancel{60} & \cancel{50} & \cancel{40} & 150 \\ \cline{1-6}
\end{tabular}
\end{enumerate}
The minimum cost is,
\begin{align}
   6\brak{10} + 3\brak{50} + 2.50 \brak{40} + 4\brak{50} = 510 
\end{align}
Hence A should supply 10,50,40 quintals to D,E and F respectively. And B should supply 50 quintals to D only.
Thus by following this we get the transportation cost to be \rupee$510$.



\end{document}